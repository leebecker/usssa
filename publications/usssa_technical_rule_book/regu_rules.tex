\section{REGU (Synchronized Team Performance) Category}
\label{sec:regu_category}

\begin{legal}
\item Competition Equipment:
    \begin{legal}
    \item Attire: \\
        A standard black Pencak Silat attire, with a white belt of 10cm which is wrapped not tied nor loosely, and without accessories. \\

        It is compulsory to have the badge of the contestant’s main association on the left chest and the badge of PERSILAT in the right chest. \\

        It is allowed to have the badge of the contestant’s main association on the left chest and PERSILAT badge on the right chest. The national flag on the left arm and the name of the country at the upper back of the attire.
    \end{legal}

\item Competition Stages

    \begin{legal}
        \item When a competition has more than 7 (seven) contestants, a pool system will be implemented.
        \item The three (3) Contestants with the highest scores from each pool will compete again in the next round. Unless the following round is the final.  The participants of the final round will be the best three (3) – in terms of gaining scores – from the previous competition pool stages.
        \item The number of pools is decided in a meeting attended by International Technical Delegates, Competition Chairperson and Council of Jury. The decision will be announced to the participants at the Technical Meeting.
        \item The pool division for contestants are determined by drawing of lots during the Technical Meeting. Voting method, i.e either manually or digitally will be decided through voting at the Technical Meeting.
        \item Each category should have a minimum of 2 (two) teams. When there are only two teams, the competition goes directly to the final round.
        \end{legal}

\item Duration of Competition \\

      The performance duration is 3 (three) minutes.

\item Competition Procedure
    \begin{legal}
    \item The beginning of competition:
        \begin{enumerate}[label=\alph*.]
        \item Juries reporting for duty to the Competition Chairperson from the right side of the 
              Competition Chairperson
        \item how respect and readiness to perform duty
        \item Taking the allocated seats
        \end{enumerate}

    \item Pesilat
        \begin{enumerate}[label=\alph*.]
        \item Entering the arena from the left side of the Competition Chairperson
        \item Walk towards the centre of arena
        \item Pesilat is to place the weapon on the mat (no assistance from coach)
        \item Show respect to the Competition Chairperson and turning back to show respect to the Jury committee.
        \end{enumerate}

    \item Competition Chairperson will signal the Juries, time keeper and other Competition official to alert them that duty is about to begin.

    \item The showcase
        \begin{enumerate}[label=\alph*.]
        \item Showcase the Opening PERSILAT greeting
        \item The gong will be stricken to mark the beginning of performance time
        \item Pesilat will perform the showcase
        \item The gong will be stricken to mark the end of performing time
        \end{enumerate}

    \item At the end of the performance
        \begin{enumerate}[label=\alph*.]
        \item Contestant to show respect to the Juries and Competition Chairperson from the center of the arena
        \item To leave the arena by the left side of the Competition Chairperson
        \end{enumerate}

    \item Time Keeping
        \begin{enumerate}[label=\alph*.]
        \item The Competition Chairperson will make sure/take charge of the showcase time
        \item The Time Keeper will keep track of the 3 minutes showcase
        \item Competition Chairperson will announce the actual showcase time. (If digital scoring is used, 
              the time tracking will be as displayed on the screen)
        \end{enumerate}

    \end{legal} % 1.1

\item Competition Rules
    \begin{legal}
    \item Rules of the game
        \begin{legal}
        \item The participants perform Jurus Wajib Regu for 3 (three) minutes. 
            A tolerance period of +/- 5 seconds is allowed.  Should the tolerance period go beyond the limit, 
            penalty will be imposed accordingly.
        \item Jurus Wajib Regu is performed according to a fixed sequence of movements and the accuracy of jurus, 
            rhythm, firmness and soulfulness designated for the jurus.
        \item Uttering of sound is allowed
        \end{legal}

    \item Penalties
        \begin{legal}
        \item \label{pt:regu_deductions} The score deduction penalty imposed due to contestants’ fault consists of:
            \begin{enumerate}[label=\alph*.]
            \item Errors in the movement and movement detail
                \begin{enumerate}[label*=\arabic*.]
                \item Deduction of 1 (one) point penalty is imposed each time contestant performs faulty movement i.e.:
                    \begin{enumerate}[label=\roman*.]
                    \item Errors in the movement details
                    \item Errors in movement sequence
                    \end{enumerate}
                \item Every mission movement (not performed)
                \item Each time a movement is not performed in `team-harmony'
                \end{enumerate}

            \item Time factor
                \begin{enumerate}[label*=\arabic*.]
                \item \label{pt:regu_time_factor} Beyond tolerance period \\
                    Ten (10) to fifteen (15) seconds – deduction of 10 points for Pre-teen and Pre-Junior 
                    categories and five (5) to ten (10) seconds for Junior, Senior and masters categories.\\

                    Should showcase go beyond these tolerance period the showcase will be stopped and 
                    declared disqualified.
                \end{enumerate}


            \item Other factors
                \begin{enumerate}[label*=\arabic*.]
                \item Deduction of 5 (five) points penalty will be imposed to the contestant each time they cross 
                    the arena borderline (10m x 10m). – To step out of the arena with even only one foot.
                \item Deduction of 5 (five) points penalty will be imposed to the contestant not properly dressed 
                    according to the rules.
                \end{enumerate}
            \end{enumerate}

        \item Walk-Out \\
            Participation will be declared as Walk-out should the contestant fail to report to the Competition’s 
            Secretary after being call for the 3rd time.\\

            The interval between the call outs will be at thirty (30) seconds each.

        \item Disqualification
            A Pesilat will be disqualified for:
            \begin{enumerate}[label=\alph*.]
            \item Wearing a non-compliant uniform 
            \item Time limit violations as stated in point~\ref{pt:regu_deductions}~\ref{pt:regu_time_factor}
            \item Being unable to show the letter of medical health before the start of competition
            \end{enumerate}
        \end{legal} % 1
    \end{legal} % 1

\item Scoring
    \begin{legal}
    \item Scoring consists of:
        \begin{legal}
        \item Accuracy score includes the following elements:
            \begin{enumerate}[label*=\alph*.]
            \item The accuracy of movement in each Jurus
            \item The accuracy of movement sequence
            \item The accuracy of jurus sequence
            \end{enumerate}
            Score obtained from the total number of movements in Jurus Wajib Regu (100 movements) deducted by the penalty points

        \item The score of harmony, firmness and solidity includes the following elements:
            \begin{enumerate}[label*=\alph*.]
            \item The harmony, firmness and solidity of movement
            \item Rhythm harmony of movement
            \item Soulfulness harmony of movement
            \item Power and stamina
            \end{enumerate}
        Score ranges from 50 (fifty) to 60 (sixty) points which is the total score of the above four elements.
        \end{legal} % 1.1.1
    \end{legal} % 1.1

\item Decision and announcement of the winner
    \begin{legal}
    \item  \label{pt:regu_hi_score} The winner is the contestant who gains the highest score for his/her performance from 3 (three) out of 5 (five) jurors with elimination of the highest and the lowest score.
    \item If the scores are equal, the winner will be determined accordingly:
            \begin{enumerate}[label*=\roman*.]
            \item The contestant who gains the total highest Technique points from the 3 (three) jurors as decided in para \ref{pt:regu_hi_score}.
            \item The contestant who gains the highest points in firmness, soulfulness and stamina from 3 (three) jurors as decided in para \ref{pt:regu_hi_score}.
            \item ) The contestant whose duration of performance is the closest to precise time of 3 (three) minutes.
            \item The winner is the contestant who gains the least penalty points.
            \item If the result remains the same, the Competition Chairperson will do a coin toss on to the mat witnessed by Technical Delegate, Council of Juror and Team Managers of respective contestant.
            \end{enumerate}

    \item The score of each contestant is announced after the Jury has finished their task in giving score to all contestant of every Jurus Regu Wajib (Team) category. Total obtained scores will be shown in scoreboard while announced by Competition Chairperson except when using digital scoreboard where the scores from each Jury and total scores are displayed in the screen instantly.
    \end{legal}
\end{legal} % 1

